\documentclass[UTF8]{ctexart}
\usepackage{listings} % 引入listings宏包
\usepackage{xcolor} % 引入xcolor宏包
\usepackage{graphicx}
\usepackage{geometry}
\geometry{left=2.5cm,right=2.5cm,top=2.5cm,bottom=2.5cm}
\usepackage{wrapfig}
\usepackage{enumitem}
\usepackage{lipsum}
\lstset{
  basicstyle=\ttfamily\small, % 使用等宽字体
  frame=single, % 代码块周围有边框
  breaklines=true, % 自动换行
  postbreak=\mbox{\textcolor{red}{$\hookrightarrow$}\space}, % 换行标记
  showstringspaces=false, % 不显示字符串中的空格
  commentstyle=\itshape, % 注释样式
  keywordstyle=\bfseries % 关键字样式
}

% 自定义命令,定义学号和标题
\newcommand{\studentnumber}{3200000000}
% 定义作者信息和学号
\author{
  \textbf{专业:}电子信息工程 \\[2ex]
  \textbf{姓名:} iiiiling\\[2ex]
  \textbf{学号:}\studentnumber \\[2ex]
  \textbf{日期:}2024.9.23\\[2ex]%日期
  \textbf{地点:}东三-202\\[2ex]%地点
}
\title{
    \begin{figure}[htbp]
        \centering
        \includegraphics[width=0.8\textwidth]{img.png}
    \end{figure}
}
% 定义作者信息和学号

\begin{document}

\maketitle

\pagestyle{empty} % 设置后续页面的样式为 'empty'
% 双栏排版

\noindent % 确保没有缩进

\begin{minipage}[t]{0.45\textwidth}
\begin{enumerate}
    \item[1.] 课程名称:电路电子技术实验
    \item[2.] 指导老师:张伟
    \item[3.] 成绩:100
    \item[4.] 实验名称: 
    \item[5.] 实验类型:
    \item[6.] 同组学生姓名:syz
\end{enumerate}
\end{minipage}%
\hfill % 添加一些空间,这一段谨慎添加空行
\begin{minipage}[t]{0.45\textwidth}
\begin{enumerate}
    \item[一.] 实验目的和要求
    \item[二.] 实验内容和原理
    \item[三.] 主要仪器设备
    \item[四.] 操作方法和实验步骤
    \item[五.] 实验数据记录和处理
    \item[六.] 实验结果与分析
    \item[七.] 讨论、心得
\end{enumerate}
\end{minipage}

\vspace{1cm}

\noindent % 确保没有缩进
\hfill % 添加一些空间
% 上面这一段可以删掉,本来要写一个目录的,但是就两页纸写什么目录。
% 但是单单左边那一列太单调了,所以拿来凑数,算是一个不合格的目录,删了无妨。


\centerline{\Large{ 实验一 :直流电压电流和电阻的测量}}

\section{实验目的}
% 如果你不喜欢这个编号,可以用*取消编号
\subsection{}
\subsection{}
\subsection{}

\section{实验内容和原理}
\subsection{}
\subsection{}
\subsection{}

\section{主要仪器设备}

\subsection{}

\subsection{}

\subsection{}

\section{实验内容}

\subsection{实验内容1的实验方法、步骤、数据记录、数据处理、实验结果分析}

\subsection{实验内容2的实验方法、步骤、数据记录、数据处理、实验结果分析}

\subsection{实验内容3的实验方法、步骤、数据记录、数据处理、实验结果分析}

\section{实验结果}

\section{讨论、心得}

% 参考

\section{参考代码}

\subsection{数学公式}

微积分:$\int_{a}^{b} f(x)dx$ \par
曲线积分:$\oint_{C} f(x,y)ds$ \par
累加:$\sum_{i=1}^{n} a_i$ \par
leq:$\leq$ \par
geq:$\geq$ \par
neq:$\neq$ \par
infty:$\infty$ \par
极限:$\lim_{x \to \infty} f(x)$ \par
数列:$a_1, a_2, a_3, \ldots, a_n$ \par
求导:$\frac{dy}{dx}$ \par
偏导:$\frac{\partial f}{\partial x}$ \par
分数:$\frac{a}{b}$ \par
开方:$\sqrt{2}$ \par


\subsection{插入代码}

\begin{lstlisting}[language=Python]
    def hello_world():
        print("Hello, World!")
\end{lstlisting}

\subsection{表格}
\begin{table}[htbp]
    \centering
    \begin{tabular}{|c|c|c|}
        \hline
        1 & 2 & 3 \\
        \hline
        4 & 5 & 6 \\
        \hline
        7 & 8 & 9 \\
        \hline
    \end{tabular}
    \caption{表格标题}
    \label{tab:1}
\end{table}
\subsection{图片}
\begin{figure}[htbp]
    \centering
    \includegraphics[width=0.5\textwidth]{img.png}
    \caption{图片标题}
    \label{fig:1}
\end{figure}

\end{document}